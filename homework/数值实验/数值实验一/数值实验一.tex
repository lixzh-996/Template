%% Numerical Experiment Report Template %%
%----------------------------------------------------%
\documentclass[a4paper,11pt]{article}

%---------------code settings------------------------%
\usepackage{listings}
\usepackage{xcolor}
\definecolor{mygreen}{rgb}{0,0.6,0}
\definecolor{mygray}{rgb}{0.5,0.5,0.5}
\definecolor{mymauve}{rgb}{0.58,0,0.82}

\lstset{ %
  backgroundcolor=\color{white},   % choose the background color; you must add \usepackage{color} or \usepackage{xcolor}
  basicstyle=\footnotesize,        % the size of the fonts that are used for the code
  breakatwhitespace=false,         % sets if automatic breaks should only happen at whitespace
  breaklines=true,                 % sets automatic line breaking
  captionpos=bl,                    % sets the caption-position to bottom
  commentstyle=\color{mygreen},    % comment style
  deletekeywords={...},            % if you want to delete keywords from the given language
  escapeinside={\%*}{*)},          % if you want to add LaTeX within your code
  extendedchars=true,              % lets you use non-ASCII characters; for 8-bits encodings only, does not work with UTF-8
  frame=single,                    % adds a frame around the code
  keepspaces=true,                 % keeps spaces in text, useful for keeping indentation of code (possibly needs columns=flexible)
  keywordstyle=\color{blue},       % keyword style
  %language=Python,                 % the language of the code
  morekeywords={*,...},            % if you want to add more keywords to the set
  numbers=left,                    % where to put the line-numbers; possible values are (none, left, right)
  numbersep=5pt,                   % how far the line-numbers are from the code
  numberstyle=\tiny\color{mygray}, % the style that is used for the line-numbers
  rulecolor=\color{black},         % if not set, the frame-color may be changed on line-breaks within not-black text (e.g. comments (green here))
  showspaces=false,                % show spaces everywhere adding particular underscores; it overrides 'showstringspaces'
  showstringspaces=false,          % underline spaces within strings only
  showtabs=false,                  % show tabs within strings adding particular underscores
  stepnumber=1,                    % the step between two line-numbers. If it's 1, each line will be numbered
  stringstyle=\color{orange},     % string literal style
  tabsize=2,                       % sets default tabsize to 2 spaces
  %title=myPython.py                   % show the filename of files included with \lstinputlisting; also try caption instead of title
}

%---------------------other package--------------&
\usepackage[T1]{fontenc}
\usepackage[utf8x]{inputenc}
\usepackage[english]{babel}
\usepackage{float}
\usepackage[colorlinks=true, allcolors=blue]{hyperref}
\usepackage[parfill]{parskip}
\usepackage[a4paper,top=2cm,bottom=3cm,left=1.5cm,right=1.5cm,marginparwidth=2cm]{geometry}
\usepackage{graphicx}
\usepackage{fancyhdr}
\usepackage{titlesec}
\usepackage{amsmath}
\usepackage{indentfirst}
\setlength{\headheight}{30pt}
\setlength{\parindent}{2em}


\begin{document}

%--------------fancyhead------------%
\pagestyle{fancy}
\fancyhead[R]{Numerical Experiment A}
\fancyhead[L]{\includegraphics[width=4.5cm]{logo/row.png}}
\fancyfoot[R]{\includegraphics[width=3cm]{logo/spst.png}}
%---------------title---------------%
\title{\textbf{\Huge{Numerical Experiment A}}}

%--------------author---------------%
\author{\textit{Xinzhi Li} \\\quad\\Student ID:~~$\boldsymbol{2022211084}$\\\quad\\ \textit{School of Physics Science and Technology, ShanghaiTech University, Shanghai 201210, China}\\\quad \\ \textit{Email address}:\quad lixzh2022@shanghaitech.edu.cn}



%---------------Logo----------------%
\begin{figure*}[t]
\centering
\includegraphics[width=1\columnwidth]{logo/row.png}
\end{figure*}

%--------------maketitle--------------&
\maketitle\thispagestyle{empty}

%------------titlesection-------------%
\titleformat{\section}[block]{\large \bfseries}{\Roman{section}.}{1em}{\centering\MakeUppercase}[]
\titlespacing{\section}{2pt}{16pt}{10pt}
\titleformat{\subsection}[block]{}{\Alph{subsection}.}{1em}{\centering}
\titlespacing{\subsection}{2pt}{8pt}{4pt}
%--------------main body--------------&
\newpage
\setcounter{page}{1}
\section{Introducion}
\textit{\textbf{Poisson' equation}} is an elliptic partial differential equation of broad utility in theoretical physics. For example, the solution to \textit{\textbf{Poisson's equation}} is the potential field caused by a given electric charge or mass density distribution; with the potential field known, one can then calculate electrostatic or gravitational field. It is a generalization of \textit{Laplace's equation}, which is aslo frequently seen in physics. However, it is quite complicated to obtain the exact solution. Sometimes, we can only get the formal solution through \textit{Green's function}. In practice, we translate the continuous equation into a series of linear equations, and then perform a numerical method to calculate the approximate solution. In the following the section, we will use three different iterations, \textbf{Jacobi}, \textbf{Gauss-Seidel} and \textbf{SOR} to sovle a concrete problem and make some analyses. 

\section{Problem}
Considering a specific \textit{Poisson' equation}:
\begin{equation}\label{eq:1}
    \begin{cases}
        -\left(\dfrac{\partial^{2}u}{\partial x^2}+\dfrac{\partial^{2}u}{\partial y^2}\right)=f(x,y), 0<x,y<1 \\
        u(0,y)=u(1,y)=u(x,0)=u(x,1)=0
    \end{cases}
\end{equation}

Set $f(x,y)=2\pi^2 \sin(\pi x)\sin(\pi y)$, then the exact solution of Eq~(\ref{eq:1}) becomes
\begin{equation}\label{eq:2}
    u^{*}(x,y)=\sin(\pi x)\sin(\pi y)
\end{equation}

Set $h=\dfrac{1}{N}, N\in \mathcal{N}^{+}, x_{i}=ih, y_{j}=jh, u_{i,j}\approx u(x_{i},y_{j}), f_{i,j}=f(x_{i},y_{j})$.
\begin{equation}
    \begin{cases}
        \left.\dfrac{\partial^{2}u}{\partial x^2}\right|_{x=x_{i},y=y_{j}}=\dfrac{u_{i+1,j}-2u_{i,j}+u_{i-1,j}}{h^2}\\
        \left.\dfrac{\partial^{2}u}{\partial y^2}\right|_{x=x_{i},y=y_{j}}=\dfrac{u_{i,j+1}-2u_{i,j}+u_{i,j-1}}{h^2}
    \end{cases}
\end{equation}

Then we can translate Eq~(\ref{eq:1}) into the linear equations:
\begin{equation}
    \begin{cases}
        -u_{i-1,j}-u_{i,j-1}+4u_{i,j}-u_{i+1,j}-u_{i,j+1}=h^2f_{i,j} \\
        u_{i,0}=u_{0,j}=u_{i,N}=u_{N,j}=0, \quad \quad i,j=1,2,\dotsb,N-1
    \end{cases}
\end{equation}

It can be also written as 
\begin{equation}\label{eq:5}
    L_{h}u^{h}=h^2f^{h}
\end{equation}

with
\begin{equation}
    u_{j}^{h}=
        \begin{array}{|c|}
            u_{1,j}\\
            u_{2,j}\\
            \vdots\\
            u_{N-1,j}
        \end{array}
    \quad
    f_{j}^{h}=
        \begin{array}{|c|}
            f_{1,j}\\
            f_{2,j}\\
            \vdots\\
            f_{N-1,j}
        \end{array}
    \quad
    u^{h}=
        \begin{array}{|c|}
            u^{h}_{1}\\
            u^{h}_{2}\\
            \vdots\\
            u^{h}_{N-1}
        \end{array}
    \quad
    f^{h}=
        \begin{array}{|c|}
            f^{h}_{1}\\
            f^{h}_{2}\\
            \vdots\\
            f^{h}_{N-1}
        \end{array}
    \nonumber
\end{equation}
\begin{equation}    
    C=
        \begin{bmatrix}
            0 & 1 & \cdots \\
            1 & 0 & 1 \\
            \vdots  & 1 & 0 & 1 \\
              &  &\ddots & \ddots & \ddots \\
              &  &  & 1 & 0 & 1 \\
              &  &  &   & 1 & 0
        \end{bmatrix}_{(N-1)\times (N-1)}
    L_{h}=
        \begin{bmatrix}
            4I-C & -I & \cdots \\
            -I & 4I-C & -I \\
            &\ddots & \ddots & \ddots \\
            &  & -I & 4I-C & -I \\
            &  &   & -I & 4I-C
        \end{bmatrix}    
\end{equation}

\section{Numerical Results}
\subsection{Experiment 1}
Set $h=0.1$ and use three iterations to obtain the solution of the Eq~(\ref{eq:5}), under the condition $\Vert u^{(k+1)}-u^{(k)}\Vert_\infty < \epsilon, \epsilon=10^{-6}$. Evaluate the norm $\Vert u^{(k+1)}-u^{*}\Vert_\infty$. For the \textbf{SOR} iteration, $\omega=1.2,1.3,1.9,0.9$. The result is shown in Table~\ref{table1}.
\begin{table}[h]
    \centering
    \caption{Iteration Results in Experiment 1}
    \label{table1}
    \begin{tabular}{|c|c|c|c|}
        \hline
        & & & \\[-6pt]
        Method&Iterations&$\Vert u^{(k+1)}-u^{*}\Vert_\infty$&Radius of convergence $\rho$ \\
        \hline
        & & & \\[-6pt]
        Jacobi iteration&217&0.008247&0.951057\\
        \hline
        & & & \\[-6pt]
        Gauss-Seidel iteration&116&0.008256&0.904508\\
        \hline
        & & & \\[-6pt]
        SOR iteration($\omega=1.2$)&79&0.008260&0.855750\\
        \hline
        & & & \\[-6pt]
        SOR iteration($\omega=1.3$)&63&0.008262&0.818687\\
        \hline
        & & & \\[-6pt]
        SOR iteration($\omega=1.9$)&126&0.008266&0.900000\\
        \hline
        & & & \\[-6pt]
        SOR iteration($\omega=0.9$)&140&0.008254&0.921804\\
        \hline
        & & & \\[-6pt]
        SOR iteration($\omega=1.6$)&29&0.008266&0.600000\\
        \hline
        & & & \\[-6pt]
        SOR iteration($\omega=1.7$)&41&0.008265&0.700000\\
        \hline
    \end{tabular}
\end{table}

\subsection{Experiment 2}
Use Jacobi iteration to solve the equation, with different steps $h=0.1,0.05,0.02,0.01$. Known that $\rho(J)=1-2\sin^{2}\dfrac{\pi h}{2}\approx 1-\dfrac{\pi^2 h^2}{2}$. The result is shown in Table~\ref{table2}.
\begin{table}[t]
    \centering
    \caption{Iteration Results in Experiment 2}
    \label{table2}
    \begin{tabular}{|c|c|c|c|}
        \hline
        & & & \\[-6pt]
        Steps&Iterations&$\Vert u^{(k+1)}-u^{*}\Vert_\infty$&Radius of convergence $\rho$ \\
        \hline
        & & & \\[-6pt]
        $h=0.1$&217&0.008247&0.951057\\
        \hline
        & & & \\[-6pt]
        $h=0.05$&762&0.001979&0.987688\\
        \hline
        & & & \\[-6pt]
        $h=0.02$&3843&0.000176&0.998027\\
        \hline
        & & & \\[-6pt]
        $h=0.01$&12566&0.001943&0.999507\\
        \hline
    \end{tabular}
\end{table}

\section{Analysis and Remark}
Table~\ref{table1} shows that, under the same accuracy requirement, the speed of Gauss-Seidel iteration is roughly twice of the Jacobi's. This is consistent with the theory for the tridiagonal matrix:
\begin{eqnarray}
    \begin{cases}
        \rho(G)\approx\rho(J)^2 \\
        R(G)=-\ln{\rho(G)}\approx-\ln{\rho(J)^2}=-2\ln{\rho(J)}=2R(J)
    \end{cases}
\end{eqnarray}

Moreover, one can find that, for the \textbf{SOR} iteration, the iterations can be much less than the Jacobi iteration. The optimal $\omega$ is about 1.6, which is very similar to the theoretical value:
\begin{equation}
    \omega_{opt}=\dfrac{2}{1+\sqrt{1-(\rho(J))^2}}
\end{equation}

Shown in Table~\ref{table2}, the iterations becomes larger and larger when the length of step goes smaller. One of reasons could be the speed of the iteration is slower with the increasing of the radius of convergence $\rho$ (since $\rho\approx 1-\dfrac{\pi^2 h^2}{2}$). The accuray increases with the smaller $h$. However, when $h=0.01$, the $\Vert u^{(k+1)}-u^{*}\Vert_\infty$ is larger than that in $h=0.02$. This is because the speed of convergence is too slow to get the more accurate result.

On the other hand, the different function in Matlab will influence the accuracy. For example, if one use $A\backslash b$ instead of $\textit{inv}(A)*b$, the iteration result will be very different. The solution may converge to another point!

In conclusion, the SOR iteration is much better than other two iterations, and one should be careful to choose an appropritae step $h$.


\section{Code}
All the experiments are performed by Matlab programs. Here is the code:

\lstinputlisting{iteration.m}

\end{document}