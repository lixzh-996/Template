%%---------------Homework Template------------------%%
%----------------------------------------------------%
\documentclass[a4paper,9pt]{article}

%---------------code settings------------------------%
\usepackage{listings}
\usepackage{xcolor}
\definecolor{mygreen}{rgb}{0,0.6,0}
\definecolor{mygray}{rgb}{0.5,0.5,0.5}
\definecolor{mymauve}{rgb}{0.58,0,0.82}

\lstset{ %
  backgroundcolor=\color{white},   % choose the background color; you must add \usepackage{color} or \usepackage{xcolor}
  basicstyle=\footnotesize,        % the size of the fonts that are used for the code
  breakatwhitespace=false,         % sets if automatic breaks should only happen at whitespace
  breaklines=true,                 % sets automatic line breaking
  captionpos=bl,                    % sets the caption-position to bottom
  commentstyle=\color{mygreen},    % comment style
  deletekeywords={...},            % if you want to delete keywords from the given language
  escapeinside={\%*}{*)},          % if you want to add LaTeX within your code
  extendedchars=true,              % lets you use non-ASCII characters; for 8-bits encodings only, does not work with UTF-8
  frame=single,                    % adds a frame around the code
  keepspaces=true,                 % keeps spaces in text, useful for keeping indentation of code (possibly needs columns=flexible)
  keywordstyle=\color{blue},       % keyword style
  %language=Python,                 % the language of the code
  morekeywords={*,...},            % if you want to add more keywords to the set
  numbers=left,                    % where to put the line-numbers; possible values are (none, left, right)
  numbersep=5pt,                   % how far the line-numbers are from the code
  numberstyle=\tiny\color{mygray}, % the style that is used for the line-numbers
  rulecolor=\color{black},         % if not set, the frame-color may be changed on line-breaks within not-black text (e.g. comments (green here))
  showspaces=false,                % show spaces everywhere adding particular underscores; it overrides 'showstringspaces'
  showstringspaces=false,          % underline spaces within strings only
  showtabs=false,                  % show tabs within strings adding particular underscores
  stepnumber=1,                    % the step between two line-numbers. If it's 1, each line will be numbered
  stringstyle=\color{orange},     % string literal style
  tabsize=2,                       % sets default tabsize to 2 spaces
  %title=myPython.py                   % show the filename of files included with \lstinputlisting; also try caption instead of title
}

%---------------------other package--------------&
\usepackage[T1]{fontenc}
\usepackage[utf8x]{inputenc}
\usepackage[english]{babel}
\usepackage{float}
\usepackage[colorlinks=true, allcolors=blue]{hyperref}
\usepackage[parfill]{parskip}
\usepackage[a4paper,top=2cm,bottom=3cm,left=1.5cm,right=1.5cm,marginparwidth=2cm]{geometry}
\usepackage{graphicx}
\usepackage{subfigure}
\usepackage{fancyhdr}
\usepackage{titlesec}
\usepackage{amsmath}
\usepackage{amssymb}
\usepackage{indentfirst}
\setlength{\headheight}{41pt}
\setlength{\parindent}{2em}


\begin{document}

%--------------fancyhead------------%
\pagestyle{fancy}
\fancyhead[R]{Classical Electrondynamics}
\fancyhead[L]{\includegraphics[width=4.5cm]{logo/row.png}}
\fancyfoot[R]{\includegraphics[width=3cm]{logo/spst.png}}
%---------------title---------------%
\title{\textbf{\Huge{Homework-4}}}

%--------------author---------------%
\author{\textit{Xinzhi Li} \\\quad\\Student ID:~~$\boldsymbol{2022211084}$\\\quad\\ \textit{School of Physics Science and Technology, ShanghaiTech University, Shanghai 201210, China}\\\quad \\ \textit{Email address}:\quad lixzh2022@shanghaitech.edu.cn}


%---------------Logo----------------%
\begin{figure*}[t]
\centering
\includegraphics[width=1\columnwidth]{logo/row.png}
\end{figure*}

%--------------maketitle--------------&
\maketitle\thispagestyle{empty}
%--------------main body--------------&
\newpage
\setcounter{page}{1}

\begin{enumerate}
  \item Consider a hard ferromagnet in a cylindrical shape, the height of the ferromagnetic cylinder is $L$, and the raius is $a$. The magnetization is $M_0$ along $z$ direction.
  \begin{enumerate}
    \item Please calculate the magnetic field $H$ and magnetic induction $B$ at all points on the $z$ axis, both outside and inside the cylinder.
    \item Plot the ratios $\frac{B}{\mu_0M_0}$ and $\frac{H}{M_0}$ on the $z$ axis as functions of $z$ for $L=5a$.
  \end{enumerate}
  \rule[0pt]{6cm}{0.05em}
  \item Consider a free space with some localized permanent magnetization distribution $\boldsymbol{M}(\boldsymbol{r})$, and the free current density in the space is zero everywhere.
  \begin{enumerate}
    \item Please prove that for such a situation, $\int d\boldsymbol{r}^3\boldsymbol{B}\cdot\boldsymbol{H}=0$.(\textcolor{blue}{Note that the system is no longer a linear medium.})
    \item Consider a spherical hard ferromagnet with magnetization $\boldsymbol{M}=M_0\hat{z}$, with radius $a$. Please calculate the magnetic fields ($H$ and $B$) both outside and inside the sphere. Please verify that the conclusion from the previous question is correct.
  \end{enumerate}
  \rule[0pt]{6cm}{0.05em}
\end{enumerate}


















\end{document}